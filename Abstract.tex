\begin{abstract}

Exascale supercomputing will embody many revolutionary changes in the
hardware and software of high-performance computing. A particularly 
pressing issue is gaining insight into the science behind the exascale
computations, because power and I/O speed constraints will fundamentally
change current visualization and analysis workflows. 
A traditional post-processing workflow involves storing simulation
results to disk and later retrieving them for visualization and data
analysis.  However, at exascale, scientists and analysts will need a range
of options for moving data to persistent storage, as the current offline or 
post-processing pipelines will not be able to capture the data 
necessary for data analysis of these extreme scale simulations.
This Milestone explores
two alternate workflows, characterized as \insitu and \intransit, and
compares them.  We find each to have its own merits and faults, and we
provide information to help pick the best option for a particular use.

\end{abstract}
