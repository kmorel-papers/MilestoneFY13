\section{Scope}
\label{sec:Scope}

\subsection{State of the Art}

When this Milestone was designed, there were three types of activities that addressed large scale data analysis: post-processing, code-specific in-situ solutions, and communication frameworks concerned with data movement.

Specialized solutions are specific instantiations of in-situ processing tuned for a particular code.  Examples include CTH's \fix{blar} and the \fix{other blar that Kwan Liu is developing with Jackie Chan}.  These are highly efficient in-situ analysis codes developed in tandem with a specific code, and represent the pathway to the most efficient in-situ solution, tuned for the data structures, the memory/compute time trade-offs and analysis needs of a specific customer in a specific domain.  As such, we expect that specialized solutions will always offer the most compact, optimized in-situ processing for the codes they were designed for.  However, applying these in-situ codes to other domains and codes requires an investment in engineering (they cannot always be easily abstracted from the target code), a trade-off in flexibility (they are often designed to achieve a specific analysis or visualization task, often one that lends itself to optimization at large scale), and a trade-off in community support and engagement.

Another option for large scale in-situ analysis is to create a capability that is intended to support a large range of data types, analysis operations and visualization modes.  A generalized engineering solution would be more flexible, but perhaps less efficient than one tuned for a specific code.  This type of library would provide analysis capabilities for a range of problems, providing a method of prototyping and iterating on many types of analysis.  Valuable or common analysis operations can then be optimized as needed for specific codes.  This is the option that is at the core of our Milestone.

The value of this contribution is to promote the investigation of a larger trade-off space, in which memory, compute time, analysis type, and storage can all be traded more flexibly to achieve a required analytics end.  By building a robust, cross-platform capability for large scale in-situ analysis, we hope to build a community that exercises different options in the trade-off space, allowing analysts and scientists the capability to explore the impact of different combinations of memory, processing time, energy, storage, analysis algorithms to achieve their results.  In particular, as we continue towards larger, ever diverse computation platforms, it is beneficial to the community to invest in analysis capabilites that provide a range of options, so that trade-off space can be easily explored.

Thus, this Milestone has promoted work on a common analysis library - Catalyst - that is now capable of scaling to ASC-sized coupled analysis runs.  A complimentary data movement capability - Nessie - was also developed to promote the trade-off between types of in-situ analysis, and help explore the optimal solutions for resource allocation, data transport, and analytics processing.

