\section{Official Milestone}
\label{sec:OfficialMilestone}

The following is the ASC milestone our work implements.

\begin{quotation}

ASC calculations produce complex datasets that are increasingly difficult to explore and understand using traditional post-processing workflows.  To advance understanding of underlying physics, uncertainties, and results of ASC codes, SNL must gather as much relevant data as possible from large simulations.  This drives SNL to couple data analysis and visualization capability with a running simulation, so that high fidelity data can be extracted and written to disk.  This Milestone evaluates two approaches for providing such a coupling:

\begin{enumerate}
\begin{item}
In-situ processing provides ``tightly-coupled'' analysis capabilities through libraries linked directly with the simulation.  SNL has collaborated on developing an in-situ capability designed for this purpose.
\end{item}
\begin{item}
In-transit processing provides ``loosely-coupled'' analysis capabilities by performing the analysis on separate processing resources.  SNL provides this capability through a ``data services'' capability designed for this purpose.
\end{item}
\end{enumerate}

SNL will engineer, test and evaluate customer-driven operations on large-scale data created by a running simulation.  The data operations will be performed by instrumented versions of both the in-situ and in-transit solutions, with the resulting performance data published and made available to the ASC community.

A program review will be conducted, and its results documented.  A report will be submitted as a record of milestone completion.

\end{quotation}

The approaches described in the milestone correspond to the workflows in
Figure~\ref{fig:Workflows:InSitu} and Figure~\ref{fig:Workflows:InTransit},
respectively.  As previously described, we implement these two approaches
using Catalyst and Nessie.

Our ``customer-driven operations'' are encapsulated in the use case given
in Section~\ref{sec:UseCase}, which was provided by Jason Wilke to 
represent a typical shock physics analysis.

The results of our instrumentation on the coupled simulation and \vda are
given in Section~\ref{sec:Results}.  These results come from experimental
runs on the Cielo supercomputer\lcite{Doerfler2011} ranging up to over 64
thousand cores.  Our measurements record the cost of coupling \vda to a
running simulation in terms of added execution time and memory overhead,
which satisfies the deliverables of the milestone.
